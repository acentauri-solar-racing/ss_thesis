% !TeX spellcheck = en_US
% !TeX encoding = UTF-8
% !TeX root = ../report.tex

\chapter{Conclusion and Outlook}
\label{chp:conclusion}
The conclusion of the report is presented in this chapter, divided into three sections: modeling, simulation, and strategy. Each section also includes an outlook on potential improvements that could be implemented in future works.


\section{Modeling}
The modeling chapter plays a crucial role in achieving one of the main objectives of this thesis: implementing a simulation. In fact, the chapter serves as the foundation for the Plant subsystem. This includes the LVD equations that connect the velocity to the forces acting on the solar car, particularly the traction force generated by the electric motor. The latter is modeled with the Willans model and drivetrain relations. The electric motor power needed to propel the car is linked through a power balance with the PV power source and the energy stored in the battery. The model considers the influence of the wind speed and car velocity, as well as the ambient temperature on the generated power with the panels. The model of the battery is chosen to be as simple and effective as possible to include all component limitations and provide the SoC.

To increase the practical relevance of the simulation results, meteorological conditions expected in October in Australia are taken into consideration. Real road data is also used to improve the accuracy of the models. Finally, the Driver is programmed to comply with the competition rules.

As this is the first work of the \enquote{Simulation and Strategy} sub-team, future iterations could reduce the number of assumptions. Moreover, the models and parameters could be tested and evaluated as components are purchased and the car is built. Additionally, participating in the first competition would provide valuable real-world data on weather conditions, car components, and road information.

Potential future improvements are listed here:
\begin{itemize}
	\item The rolling friction coefficient is assumed constant, but as in~\cite{winningSolarCar2003book, SolUTra:2006mt} the influence of velocity could also be considered: $c_\mathrm{roll} = c_\mathrm{roll}(v)$;
	\item The density of air contributes to the aerodynamic friction loss and could be modeled to be a function of height and ambient temperature, as in~\cite{SolUTra:2006mt};
	\item The PV model could be further improved by considering the influence on the generated power by the inclination of the panels, the geographic location of the car, and the curvature of the car deck~\cite{optimalEnergyManagement:2000book};
	\item The irradiance data could be revised and divided into direct, diffusive, and reflected as in~\cite{optimalEnergyManagement:2000book}. Additionally, they could be modeled on both space and time. However, the strategy derivation presented in this report would not be possible, since it is assumed that the global irradiance is a function of time only;
	\item The simple Willans model could be extended using a more precise efficiency map that provides the motor efficiency at a given torque and angular speed;
	\item Due to high temperatures reached in the hot sunny days of the competition, the temperature degradation of the battery could play an important role in the simulation and could be crucial for the safety of the driver;
	\item Although on the highway, the speed limit data towards the end of the race showed a relative constant value of $\unitfrac[90]{km}{h}$. This value should be checked;
	\item The ambient temperature data could be corrected, since they are referred to 10 meters above the ground.
\end{itemize}


\section{Simulation}
Implementing a working simulation is beneficial in making design decisions based on quantitative results, rather than relying on intuition. Although the sensitivity analysis is dependent on the tuning parameters of the Driver, the relative difference between the design parameters helps to identify where time and effort should be directed. As previously mentioned, the Plant subsystem is fully based on the models derived. On the other hand, the Driver is developed from scratch with the goal of providing the simulation with a Decision Maker that reflects how a real driver would behave during the competition. This allows the simulation to act as a virtual testing ground for different strategies, where it is possible to make adjustments before the competition. 

However, there are also opportunities for future improvements in this area:
\begin{itemize}
	\item The Decision Maker currently does not decide to stop if the amount of energy is not sufficient to maintain a velocity of $\unitfrac[60]{km}{h}$ on open roads;
	\item The overlapping of a control stop with an overnight stop is not correctly implemented. In fact, at the moment, the control stop is spent during the night and not on the following morning;
	\item Although only for less than a second, the car velocity becomes negative at the beginning of the simulation, as the system has an unstable zero;
	\item The amount of solar energy gathered during the control stops could vary if it is possible to tilt the panels towards the sun. Therefore, it is necessary to check this possibility;
	\item Extreme weather conditions could also be simulated and tested to create a sense of how the car would respond in such cases.
\end{itemize}


\section{Strategy}
The second objective of the thesis is to develop a strategy for the entire competition. The goal is to minimize the driving time by finding the mean velocity that will ensure the available energy budget is completely used exactly at the end of the race. Both overnight stops in the time domain and control stops in the space domain are considered, as well as their possible overlapping. Furthermore, the optimal resulting SoC profile is adjusted to avoid violations of the lower SoC bound. Despite the assumptions made, the method used for this task proves to be powerful and simple in providing indicative mean velocities to aim for during the competition. However, due to time constraints, the final recursive algorithm is not finished.

Given the importance of strategy in this competition, more resources and effort could be allocated in future iterations. Opportunities for improvement exist at all three levels, from short-term to long-term strategy.

A list of potential areas of focus includes:
\begin{itemize}
	\item Correcting and completing the recursive algorithm by integrating all possible extreme cases, namely by considering the intersection between energy budget and energy loss during an overnight stop, that currently results in incorrect driving time solutions. Additionally, implement the recursion for multiple consecutive violations of the lower SoC bound as presented in~\cref{sec:strategyRecursiveAlgo};
	\item An assumption made is that the violations of the SoC could only occur at 17:00 in time domain. However, the SoC could show values below the lower bound at any point of the competition. This point could be implemented to increase the generality and reliability of the algorithm;
	\item As for the modeling section, once the car components are purchased and tested, a more precise value of the inefficiency coefficient can be estimated, resulting in a more accurate SoC profile;
	\item Investigating the use of dynamic programming or other optimization techniques to find the global optimal strategy could be a valuable future work. A comparison between these techniques and the one derived in this report could help to understand how the assumptions taken here influence the results compared to standard methods. Additionally, some algorithms allow for the use of stochastic variables, which in this case, could be used for developing a link between offline and online optimization. This type of approach is particularly useful for systems that are subject to uncertainty, such as weather conditions, traffic, and other variables that can affect the performance of the solar car during the race. It could provide a more robust and adaptable strategy for the team in the long run.
\end{itemize}
